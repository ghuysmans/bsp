\documentclass{article}
\usepackage[a4paper, top=3cm]{geometry}
\usepackage{titling}
\setlength{\droptitle}{-60pt}
\usepackage[utf8]{inputenc}
\usepackage[T1]{fontenc}
\usepackage{hyperref}
\usepackage{url}
\usepackage[english,frenchb]{babel}
\usepackage{csquotes}
\newtheorem{prop}{Propriété}
\newtheorem{dfn}{Définition}
\usepackage{amssymb}
\usepackage{amsmath}
\usepackage{cancel}
\newcommand{\bigO}{\mathcal{O}}
\newcommand{\esN}{\mathbb{N}}
\newcommand{\esR}{\mathbb{R}}
\newcommand{\reg}{\mathcal{R}}
\newcommand{\es}{\emptyset}
\newcommand{\inc}{\subseteq}
\newcommand{\sm}{\setminus}
\title{Projet de Structure de données II - Rapport final}
\author{Guillaume \textsc{Huysmans}}
\date{15 avril 2016}
\hypersetup{
	pdftitle={Projet de Structure de données II - Rapport final},
	pdfauthor={Guillaume \textsc{Huysmans}},
	pdfsubject={BSP, plan, binary space partition, Java, implementation},
	pdfkeywords={sdd2}}
\begin{document}
\maketitle
%\tableofcontents
%\newpage


\section{Arbre BSP}
Un arbre BSP (\textit{Binary Space Partition}) permet d'accéder efficacement
à des objets dans l'espace ou ici, le plan. Son principe est dérivé de celui
des arbres binaires. Chaque noeud interne représente une région $\reg$
du plan et est divisé en trois parties selon
une droite $D\equiv f(P)=aP_x+bP_y+c=0$ :
\begin{itemize}
	\item $d^-=\left\{f(P)<0|P\in\reg\right\}$ (sous-arbre gauche)
	\item une liste de segments $s \inc D$
	\item $d^+=\left\{f(P)>0|P\in\reg\right\}$ (sous-arbre droit)
\end{itemize}

À une feuille de cet arbre, aucune droite ne sera associée et elle ne
contiendra qu'une liste de segments.


\section{Implémentation}
%fonctionnement des différentes structures et algorithmes utilisés
Les calculs de projections ont été expérimentés dans
GeoGebra\footnote{\url{https://www.geogebra.org/?lang=fr}}
avant d'être implémentés en Java.


\section{Mode d'emploi}
\subsection{TestUI}
\subsection{TestCompare}
Passer par \texttt{rlwrap -c}.


\section{Résultats}
%donner et commenter les résultats de vos comparaison des heuristiques


\section{Complexité de l'algorithme du peintre}
TODO

Il est prouvé dans l'ouvrage de référence (p. 258) que
le nombre de fragments obtenus après construction d'un BSP
est en $\bigO(n+2n\ln n)=\bigO(n\ln n)$.


\section{Conclusion}
%apports, difficultés, comparaison théorie/pratique
découverte d'un algorithme et d'une structure de données élégants
problèmes de précision lors des calculs
généralisation 3D intéressante mais connaissances math suffisantes ?
	Quake et ses dérivés nomment leurs cartes ainsi et ce n'est pas un hasard.
StreamTokenizer inconnu jusqu'ici


\end{document}
