\documentclass[12pt,twocolumn]{article}
\usepackage{fullpage}
\usepackage[a4paper, margin=1.5cm, bottom=3cm]{geometry}
\usepackage{graphicx}
\usepackage{fourier} %font
\usepackage{url} %links
\usepackage{hyperref} %PDF links
\usepackage[T1]{fontenc}
\usepackage{ucs}
\usepackage[francais]{babel}
\usepackage[utf8x]{inputenc}
\newcommand{\es}{\emptyset}
\newcommand{\inc}{\subseteq}
\newcommand{\sm}{\setminus}
\usepackage{amsmath}
\usepackage{amssymb}
\usepackage{array}
\newcommand{\esN}{\mathbb{N}}
\newcommand{\esZ}{\mathbb{Z}}
\newcommand{\esR}{\mathbb{R}}
\newcommand\wb[1]{\overline{#1}}
\newcommand\xor{\oplus}
\newcommand{\bigO}{\mathcal{O}}
\newcommand{\reg}{\mathcal{R}}
\usepackage[french,onelanguage]{algorithm2e}

\title{Projet de Structure de données II - Exercice préliminaire}
\author{Guillaume \textsc{Huysmans}, Alexis \textsc{Lecocq}}
\date{31 janvier 2016}
\hypersetup{pdfauthor={Guillaume Huysmans, Alexis Lecocq},
	pdftitle={Projet de Structure de données II - Exercice préliminaire}
	pdfsubject={BSP, plan, binary space partition},
	pdfkeywords={sdd2}}
\begin{document}
\maketitle

\section{Rappels théoriques}
Un arbre BSP (\textit{Binary Space Partition}) permet d'accéder efficacement
à des objets dans l'espace ou ici, le plan. Son principe est dérivé de celui
des arbres binaires. Chaque noeud interne représente une région $\reg$
du plan et est divisé en trois parties selon
une droite $D\equiv f(P)=aP_x+bP_y+c=0$ :
\begin{itemize}
	\item $d^-=\left\{f(P)<0|P\in\reg\right\}$ (sous-arbre gauche)
	\item une liste chaînée de segments $s \inc D$
	\item $d^+=\left\{f(P)>0|P\in\reg\right\}$ (sous-arbre droit)
\end{itemize}
À une feuille de cet arbre, aucune droite ne sera associée et elle ne
contiendra qu'une liste chaînée de segments.

\section{Recherche dans une liste simplement chaînée}
La recherche dans la liste chaînée est implémentée par récursion terminale :
les deux cas de base sont la liste vide et celle dont le premier élément
est celui que l'on recherche. Le cas récursif consiste en un simple rappel
avec la suite de la liste.

\begin{algorithm}
\caption{in\_ll}
\SetAlgoLined\DontPrintSemicolon
\KwData{ll, x}
\KwResult{x $\in$ ll}
\uIf{ll = $\es$} {
	\Return{0}
}
\uElseIf{ll.data = x} {
	\Return{1}
}
\Else {
	\Return{in\_ll(ll.next, x)}
}
~
\end{algorithm}

Complexité (reprise du cours) :
\begin{itemize}
	\item Dans le pire des cas : $\bigO(n)$
	\item En moyenne : $\bigO(n/2)$
\end{itemize}

\section{Recherche dans un BSP}
La recherche dans un BSP est analogue à celle dans un ABR.
Pour déterminer si un segment $[A,B]$ est totalement dans un demi-plan
délimité par $f(P)=aP_x+bP_y+c=0$, il suffit que $f(A) \times f(B) \geq 0$.
La division du segment $[A,B]$ est assez facilement implémentée :
on calcule son l'intersection $X$ avec le segment qui détermine
les demi-plans puis on le décompose en deux autres : $[A,X]$ et $[X,B]$.

\begin{algorithm}
\caption{contains}
\SetAlgoLined\DontPrintSemicolon
\KwData{bst, segment}
\KwResult{segment $\in$ bst}
\uIf{bst est une feuille} {
	\Return{in\_ll(segment, bst.list)}
}
\uElseIf{segment $\inc d^-$} {
	\Return{contains($bst^-$, segment)}
}
\uElseIf{segment $\inc d^+$} {
	\Return{contains($bst^+$, segment)}
}
\Else {
	$segment^-$ = $segment \cap d^-$ \;
	$segment^+$ = $segment \cap d^+$ \;
	\Return{contains($bst^-$, $segment^-$) $\land$
		contains($bst^+$, $segment^+$)}
}
~
\end{algorithm}

Il est prouvé dans l'ouvrage de référence (p. 258) que
le nombre de fragments obtenus après construction d'un BSP
est en $\bigO(n+2n\ln n)=\bigO(n\ln n)$.

Par noeud interne, en ignorant les appels récursifs,
on effectue un travail en $\bigO(1)$ :
il n'y a que des tests et calculs simples.
Par feuille, au pire, une recherche en $\bigO(n\ln n)$ est effectuée.

Dans le cas d'une autopartition, il y aura au pire $n$ noeuds internes.

Si une seule feuille est traitée, on obtient
$1 \times \bigO(n\ln n) + n \times \bigO(1) = \bigO(n\ln n)$.

\end{document}
